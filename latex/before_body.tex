% you may need to leave a few empty pages before the dedication page

%\cleardoublepage\newpage\thispagestyle{empty}\null
%\cleardoublepage\newpage\thispagestyle{empty}\null

\cleardoublepage\newpage

\thispagestyle{empty}

\graphicspath{{images/}}

\chapter*{Preface}\label{preface}

\addcontentsline{toc}{chapter}{Preface}

\noindent For the course "R for Statistics" (\href{https://wur.osiris-student.nl/#/onderwijscatalogus/extern/cursus?cursuscode=MAT50303&collegejaar=huidig}{MAT50303}) it is assumed that you have acquired knowledge about statistics. The minimal required level is "Advanced Statistics" (AS, \href{https://wur.osiris-student.nl/#/onderwijscatalogus/extern/cursus?cursuscode=MAT20306&collegejaar=huidig}{MAT20306}), "Quantitative Research Methodology and Statistics" (QRMS, \href{https://wur.osiris-student.nl/#/onderwijscatalogus/extern/cursus?cursuscode=MAT22306&collegejaar=huidig}{MAT22306}), or the course "Advanced Statistics for Nutritionists" (ASN, \href{https://wur.osiris-student.nl/#/onderwijscatalogus/extern/cursus?cursuscode=MAT24306&collegejaar=huidig}{MAT24306}). If you do not have sufficient pre-knowledge, or your statistical courses were so long ago that memory traces have evaporated, you are strongly advised to follow a statistics course first. AS, ASN and QRMS are presented in English.

The aim of "R for Statistics" is to provide an introduction to R, a computer language and environment for statistics and graphics. The course will focus on getting familiar with the R environment by means of RStudio, to use R for manipulation and exploration of data and to perform all statistical analyses learned in the basic and advanced statistics courses, i.e. simple linear and multiple linear regression models, analysis of variance for Completely Randomized Designs (CRD), Randomized Complete Block Designs (RCBD) and factorial designs, ANalysis of COVAriance (ANCOVA), non-parametric methods and analysis of categorical data. The correct interpretation and explanation of the results of preformed statistical analyses is expected based on the assumed prior knowledge. Basic programming, visualization, as well as reproducibility of results in R will also be part of this course.

Students with an interest in statistics may benefit from the courses "Modern Statistics for the Life Sciences" (MSLS, \href{https://wur.osiris-student.nl/#/onderwijscatalogus/extern/cursus?cursuscode=ABG30806&collegejaar=huidig}{ABG30806}), "Statistics for Data Scientists" (\href{https://wur.osiris-student.nl/#/onderwijscatalogus/extern/cursus?cursuscode=MAT32806&collegejaar=huidig}{MAT32806}), "Data Science for Plant Breeding and Genetics" (\href{https://wur.osiris-student.nl/#/onderwijscatalogus/extern/cursus?cursuscode=MAT33306&collegejaar=huidig}{MAT33306}) and/or "Bayesian Data Analysis" (\href{https://wur.osiris-student.nl/#/onderwijscatalogus/extern/cursus?cursuscode=MAT34806&collegejaar=huidig}{MAT34806}) as a follow-up. All these courses are presented in English and build upon AS, or ASN. In MSLS logistic and log linear models (for binary and count data), variance components models (for dependent data, both balanced and unbalanced), general inference techniques (maximum likelihood estimation, Wald test, likelihood ratio test), posterior Bayesian inference using Markov Chain Monte Carlo, and applications in genetics and epidemiology are discussed. Apart from these courses students are also welcome at Biometris for a MSc-thesis Mathematical and Statistical Methods (MAT80418, MAT80439).

\begin{figure}[!h]%[!htbp]
  \begin{center}%
    \includegraphics[width=11.6cm]{biometris.png}%
    \label{fig:biometris}%
  \end{center}%
\end{figure}%

The aforementioned courses are organized by or in cooperation with Biometris. \href{https://www.wur.eu/biometris}{Biometris} is part of Wageningen University \& Research and is responsible for a number of short courses for PhD students and participates in many other courses as well. Biometris is a partner in the M.Sc. course in Applied Statistics at Leiden University. Staff members from Biometris are actively involved in research, mainly in statistical genetics, systems biology and ecology, food health and safety, omics, and big data. We have, however, an interest in other topics as well. If you have an interest in statistics, or work on your master thesis will involve use of advanced statistics, you might consider doing your thesis in cooperation with Biometris. In that case please contact a member of staff of Biometris about this.

% \setlength{\abovedisplayskip}{-5pt}
% \setlength{\abovedisplayshortskip}{-5pt}

\cleardoublepage\newpage
